
\section{Considerações sobre o aprendizado nesse projeto}

Neste projeto foi possível compreender de maneira detalhada como os processadores executam os programas desenvolvidos em linguagem de alto nível a partir das instruções geradas pelo compilador. Quando se cria software e sistemas de informação, essa camada do software localizada muito próxima do hardware se torna completamente oculta aos desenvolvedores e é exatamente nesse contexto que o simulador do processador atua, decodificando e executando cada instrução oriunda de processos anteriores de compilação e montagem (\textit{linked}).

Conhecer em detalhes uma Arquitetura do Conjunto de Instruções (\textit{Instruction Set Architecture - ISA}) RISC-V foi bem interessante, demonstrando o quão simples pode ser uma instrução executada por um processador e ao mesmo tempo poderosa, dando suporte às linguagens de programação de alto nível.

As convenções utilizadas na especificação das instruções RISC-V foi provavelmente o maior desafio encontrado no desenvolvimento deste projeto. A documentação da implementação de determinadas instruções presume conhecimentos prévios e, por isso, a dificuldade no entendimento.

Por fim, destaco o aprendizado relativo à arquitetura do conjunto de instruções do processador RISC-V associado aos conteúdos da disciplina apresentados durante as aulas. Além disso, o objetivo proposto pelo projeto foi alcançado, pois todas as instruções RISC-V RV32IM foram implementadas e os programas testes executados, conforme pode-se verificar nos arquivos log do simulador.
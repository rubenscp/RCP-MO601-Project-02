
\section{Considerações sobre o aprendizado nesse projeto}

Neste projeto foi possível comprender em detalhes como os processadores executam os programas desenvolvidos em linguagem de alto nível a partir das instruções geradas pelo compilador. Quando se cria software e sistemas de informação, essa camada do software muito próxima do hardware se torna completamente oculta aos desenvolvedores de sistemas de informação e software e é exatamente nesse contexto que o simulador do processador atua, avaliando (decodificando) e executando cada instrução oriunda de processos anteriores de compilação e montagem (\textit{linked}).

Conhecer em detalhes uma Arquitetura do Conjunto de Instruções (\textit{Instruction Set Architecture (ISA)}) RISC-V foi bem interessante, demonstrando o quão simples pode ser uma instrução executada por um processador e ao mesmo tempo poderosa dando suporte às linguagens de programação de alto nível.

Os resultados alcançados por esse trabalho foram  



------------------------------------ ---------------- 

O assunto tratado nesse projeto é muito diferente dos temas que venho trabalhando nas últimas três décadas
de desenvolvimento de sistemas de informação. Contudo, desafios são excelentes oportunidades de aprendizado, 
fixação de novos conceitos e exploração de tecnologias e ferramentas.

Neste projeto foi possível apreender conhecimentos principalmente relacionados à área de arquitetura de computadores 
com destaque ao conceito de atraso em circuitos lógicos e funcionamento das portas lógicas,
elementos muitos simples, mas combinadas entre si podem produzir resultados interessantes e importantes.
A compreensão do conceito de atraso no circuito lógico aplicado às portas lógicas 
ao mesmo tempo que os sinais de entrada das variáveis são inseridos, representou 
o ponto mais relevante deste trabalho. Embora houveram dificuldades no início, mas 
após estudos e diálogos com o professor, foi possível compreender a ideia do atraso e, sabe-se  que quando entendemos 
a definição de um problema, boa parte de sua resolução está encaminhada.

Outro ponto de destaque é o uso de novas ferramentas no desenovlvimento e entrega do trabalho combinando os produtos 
de GitHub e Docker, 
atualmente ferramentas de ampla utilização não somente no meio acadêmico mas no âmbito organizacional e empresarial.

Ressalto dois pontos importantes para o sucesso do trabalho: 1) a especificação do projeto está muito bem detalhada 
permitindo o pleno entendimento do trabalho a ser desenvolvido e das entregas esperadas; e 2) elevada disponibilidade 
do professor no atendimento aos alunos.

Por fim, destaco que os resultados alcançados nesse projeto são fruto da combinação de uma completa especificação 
do projeto, boa disponibilidade do professor em esclarecer e colaborar com o estudante,
elevada dedicação no desenvolvimento do projeto conforme especificado e foco na entrega do produto dentro do prazo
esperado.


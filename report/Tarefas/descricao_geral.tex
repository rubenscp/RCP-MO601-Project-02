
\section{Descrição geral}


O objetivo deste projeto é implementar um simulador básico de circuitos lógicos considerando portas lógicas, 
estímulos de entrada e tipo de atraso em cada simulação.

As portas lógicas possíveis nesse simulador são AND, OR, NOT, NAND, NOR e XOR. 
Os estímulos podem ser variáveis ou indicadores de tempo.
As variáveis indicam os valores dos sinais de entrada (0 ou 1)  
durante a simulação do circuito lógico, sendo permitida até 24 variáveis distintas identificadas 
pelas letras do alfabeto do idioma português. Os indicadores de tempo são identificados com o caractere '+' \space
seguido de um número inteiro qualquer \textit{n}, significando que deve-se avançar \textit{n}
tempos de simulação.

O comportamento do simulador considera dois tipos de atraso: a) atraso 0 (\textit{delay 0}) 
indicando que as portas lógicas operam no mesmo tempo da simulação em curso; ou b) atraso 1 (\textit{delay 1}) onde as portas 
lógicas operarão com atraso de \textbf{um} tempo de simulação e, portanto, os sinais de entrada nas portas lógicas 
serão aqueles sinais do tempo 
imediatamente anterior ao tempo atual da simulação.

Para cada unidade de tempo da simulação, deve-se emitir uma saída com uma lista contendo o valor do tempo da simulação
e os valores de cada variável.

Para o encerramento de uma simulação específica, será considerada como condição de parada a produção de 
duas linhas de saída exatamente 
iguais, indicando que o circuito não possui estímulos a processar e que todas as portas lógicas também processaram seus 
sinais levando em conta o tipo de atraso do circuito. O encerramento da simulação significa que o circuito 
estabilizou e que não haverá nenhuma nova saída diferente das anteriores.

Para testar o simulador, vários cenários de teste são fornecidos contendo diferentes circuitos lógicos e estímulos.
Para cada cenário de teste são executadas as principais etapas abaixo:
\begin{enumerate}
    \item Criação do circuito com suas respectivas portas lógicas.
    \item Processamento de todos os estímulos com atraso 0 (sem atraso).
    \item Geração da saída da simulação com atraso 0.
    \item Processamento de todos os estímulos com atraso 1.
    \item Geração da saída da simulação com atraso 1.
\end{enumerate}
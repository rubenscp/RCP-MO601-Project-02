
\section{Descrição geral}


O objetivo deste projeto é implementar um simulador do processador RISC-V RV32IM que executa o conjunto das instruçãos básicas de inteiros, as instruções de multiplicação e de divisão inteiras, em uma arquitetura de 32 bits.


O projeto utiliza como entrada os programas codificados na linguagem C disponibilizados no repositório ACStone (\textit{https://github.com/rjazevedo/ACStone}), que possuem funcionalidades e utilidades reduzidas, contudo são excelentes artefatos para uso em simuladores de processadores. Nesse repositório foram disponibilizados 77 programas fonte C de testes os quais contemplam todas as instruções implementadas neste simulador. O conjunto de instruções RISC-V é composto de 47 instruções básicas (\textit{RV32I Base Instruction Set}) e 8 instruções de multplicaçao e divisão (\textit{RV32M Standard Extension}), totalizando 55 instruções. 

Os programas fonte foram compilados utilizando-se a arquitetura 32 bits, em seguida montados em formato assembler (\textit{linked}) e utilizados como entrada no simulador do processador RISC-V.

O simulador implementa um modelo de processador contendo as estruturas de dados básicas com destaque à memória RAM com tamanho suficiente para acomodar as instruções e estuturas de dados, 32 registradores e contador de programa (PC - \textit{Program Counter}). Cada programa assembler foi processado separadamente produzindo um arquivo de saída com o log da execução conforme formato especificado na definição desse projeto (padrão ABI). Cada arquivo de log é a evidencia da simulação de um programa fonte/assembler.


\section{Descrição geral}


O objetivo deste projeto é implementar um simulador do processador RISC-V RV32IM que executa o conjunto das instruçãos básicas 32 bits de inteiros, de multiplicação e de divisão de inteiros.

O projeto utiliza como entrada os programas codificados na linguagem C disponibilizados no repositório ACStone (\textit{https://github.com/rjazevedo/ACStone}), que possuem funcionalidades e utilidades reduzidas, contudo são excelentes ferramentas para o uso em simuladores de instruções do processador RISCV. Nesse repositório foram disponibilizados 77 programas C os quais contemplam o total de instruções utilizadas neste simulador. O conjunto de instruções RISC-V possui 47 instruções básicas (\textit{RV31I Base Instruction Set}) e 8 instruções de multplicaçao e divisão (\textit{RV32M Standard Extension}), totalizando 55 instruções. 

Os programas fonte C foram compilados utilizando-se a arquitetura 32 bits (RV32IM), em seguida montados em formato assembler (\textit{linked}) e utilizados como entrada no simulador do processador RISCV.

O simulador implementa um modelo de processador contendo as estruturas de dados básicascom destaque à memória RAM com tamanho suficiente para acomodar as instruções geradas dos programas compilados e as estuturas de dados maipuladas por esses programas, 32 registradores e contador de programa (PC - \textit{Programa Counter}). Cada programa foi processado separadamente e um arquivo de log foi gerado conforme o formato especicado na definição desse projeto e o padrão ABI, sendo que cada arquivo de saída representa a evidencia da simulação de um programa fonte.


\section{Descrição geral}


O objetivo deste projeto é implementar um simulador do processador RISC-V RV32IM que deverá executar o conjunto das instruçãos básicas de inteiros e as instruções de multiplicação e divisão inteiras considerando 32 bits.

O projeto deverá utilizar os programas codificados na linguagem C disponibilizados no repositório ACStone (\textit{https://github.com/rjazevedo/ACStone}), os quais possuem funcionalidade e utilidade reduzidas, contudo, são excelentes para o uso em simuladores de instruções do processador RISCV. Nesse repositório foram disponibilizados 77 programas em C os quais contemplam o total de instruções utilizadas no simulador. O conjunto de instruções RISC-V possui 47 instruções básicas (\textit{RV31I Base Instruction Set}) e 8 instruções de multplicaçao e divisão (\textit{RV32M Standard Extension}), totalizando 55 instruções. 

Os programas fonte na linguagem C foram compilados utilizando-se a arquitetura 32 bits (RV32IM) e, posteriormente, gerou-se o \textit{dump} de cada programa compilado com todas as instruções em formato assembler, os quais foram utilizados como entrada para o simulador do processador.

O simulador implementou um modelo de processador contendo estruturas de dados que representaram a memória RAM com tamanho suficiente para acomodar as instruções geradas e as estuturas de dados maipuladas nos programas, 32 registradores e o contador de programa (PC - \textit{Programa Counter}). Cada programa foi processado e um arquivo de log foi gerado conforme o formato especicado na definição desse projeto e o padrão ABI, sendo que cada arquivo representa a evidencia da simulação de um programa fonte.
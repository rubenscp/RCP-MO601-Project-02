
\section{Testes}

A execução dos testes compreende duas etapas: 1) geração dos programas em formato assembler (\textit{linked}); e 2) simulação da execução dos programas assembler de acordo com o processador RISC-V RV32IM.

A etapa 1 é realizada a partir dos programas previamente compilados e fornecidos pelo professor (extensão "riscv"), localizados em \textit{https://drive.google.com/drive/u/2/folders/1ei8E-qk2dwQvCWJv8 ovGJiIdLjw7yVfP}. Com isso, a etapa de compilação foi previamente realizada bastando criar os programas assembler por meio da execução do arquivo em lote \textit{dump-programas-compilados.bat}. Este \textit{script} executa o utilitário \textit{riscv32-unknown-elf-objdump.exe} gerando os programas assembler (extensão "asm") a partir dos programas compilados.

A etapa 2 realiza a simulação do processador RISC-V executando cada programa assembler e gerando um arquivo de log (extensão "log") para cada programa simulado, a fim de evidenciar a sua execução.

Os arquivos foram organizados em uma pasta de testes específica denominada \textit{test} e localizada dentro da pasta do projeto \textit{RCP-MO601-Project-02}. A pasta \textit{test} contém todos os programas fonte em C (extensão "c"), os programas compilados previamente e os programas assembler gerados na etapa 1. Para facilitar a visualização do resultado do simulador, foi criada a pasta \textit{log} dentro da pasta \textit{test} a fim de acomodar os arquivos de saída da simulação.

Ao final da simulação, as estatísticas da simulação estarão disponíveis no arquivo \textit{\textunderscore riscv\textunderscore simulator \textunderscore statistic.txt} na pasta \textit{log}.

\section{Testes}

A execução dos testes compreende duas etapas: 1) geração dos programas em formato assembler (\textit{linked}); e 2) simulação da execução dos programas assembler de acordo com o processador RISC-V RV32IM.

A etapa 1 é realizada a partir dos programas previamente compilados compilados (extensão "riscv") e fornecidos pelo professor, os quais estão localizados em \textit{https://drive.google.com/drive/u/2/folders /1ei8E-qk2dwQvCWJv8ovGJiIdLjw7yVfP}. Devido a isso, a etapa de compilação já foi sanada previamente bastando criar os programas assembler. Para isso, uUm arquivo em lote \textit{dump-programas-compilados.bat} deve ser executado para que o programa utilitário \textit{riscv32-unknown-elf-objdump.exe} crie os programas em assembler (extensão "asm") a partir dos programas compilados (extensão "riscv"). 

A etapa 2 realiza a simulação do processador RISC-V executando cada programa assembler e gerando um arquivo de log (extensão "log") para cada programa simulado, a fim de evidenciar a sua execução.

Os arquivos foram organizados em uma pasta de testes específica denominada \textit{test} e localizada dentro da pasta do projeto (\textit{RCP-MO601-Project-02}). A pasta \textit{test} contém todos os programas fonte em C (extensão "c"), os programas compilados previamente fornecidos pelo professor e os programas assembler gerados na etapa 1. Para facilitar a visualização do resultado do simulador, foi criada a pasta \textit{log} dentro da pasta \textit{test} a fim de acomodar os arquivos de saída da simulação.

A execução de simulação dos programas assembler apresenta no terminal ou console a identificação do programa e a contagem total dos ciclos de execução das instruções.
